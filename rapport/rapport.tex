{\rtf1\ansi\ansicpg1252\cocoartf1561\cocoasubrtf400
{\fonttbl\f0\fswiss\fcharset0 Helvetica;}
{\colortbl;\red255\green255\blue255;}
{\*\expandedcolortbl;;}
\paperw11900\paperh16840\margl1440\margr1440\vieww10800\viewh8400\viewkind0
\pard\tx566\tx1133\tx1700\tx2267\tx2834\tx3401\tx3968\tx4535\tx5102\tx5669\tx6236\tx6803\pardirnatural\partightenfactor0

\f0\fs24 \cf0 \\documentclass\{article\}\
\
% If you're new to LaTeX, here's some short tutorials:\
% https://www.overleaf.com/learn/latex/Learn_LaTeX_in_30_minutes\
% https://en.wikibooks.org/wiki/LaTeX/Basics\
\
% Formatting\
\\usepackage[utf8]\{inputenc\}\
\\usepackage[margin=1in]\{geometry\}\
\\usepackage[titletoc,title]\{appendix\}\
\
% Math\
% https://www.overleaf.com/learn/latex/Mathematical_expressions\
% https://en.wikibooks.org/wiki/LaTeX/Mathematics\
\\usepackage\{amsmath,amsfonts,amssymb,mathtools\}\
\
% Images\
% https://www.overleaf.com/learn/latex/Inserting_Images\
% https://en.wikibooks.org/wiki/LaTeX/Floats,_Figures_and_Captions\
\\usepackage\{graphicx,float\}\
\
% Tables\
% https://www.overleaf.com/learn/latex/Tables\
% https://en.wikibooks.org/wiki/LaTeX/Tables\
\
% Algorithms\
% https://www.overleaf.com/learn/latex/algorithms\
% https://en.wikibooks.org/wiki/LaTeX/Algorithms\
\\usepackage[ruled,vlined]\{algorithm2e\}\
\\usepackage\{algorithmic\}\
\
% Code syntax highlighting\
% https://www.overleaf.com/learn/latex/Code_Highlighting_with_minted\
\\usepackage\{minted\}\
\\usemintedstyle\{borland\}\
\
% References\
% https://www.overleaf.com/learn/latex/Bibliography_management_in_LaTeX\
% https://en.wikibooks.org/wiki/LaTeX/Bibliography_Management\
\\usepackage\{biblatex\}\
\\addbibresource\{references.bib\}\
\
% Title content\
\\title\{LO21 - Syst\'e8me expert\}\
\\author\{Guillaume RUFF & Driss KIHAL\}\
\\date\{Automne 2020\}\
\
\\begin\{document\}\
\
\\maketitle\
\
% Introduction and Overview\
\\section\{Introduction\}\
Un syst\'e8me expert est compos\'e9 d'une base de conaissances, une base de faits et un moteur d'inf\'e9rence.\
C'est un outil capable de reproduire les m\'e9canismes cognitifs d'un expert, dans un domaine particulier.\
\
\
%  Theoretical Background\
\\section\{Base de connaissances\}\
Une base connaissances est une liste de r\'e8gles on ne fait aucune supposition sur leurs v\'e9racit\'e9 en dehors du moteur d'inf\'e9rence et de la base de faits. \
\\subsection\{R\'e8gle\}\
On d\'e9fini une r\'e8gle comme une liste de propositions divis\'e9es en 2 parties la pr\'e9misse et la conclusion. \
\\\\Soit E l'ensemble des propositions de l'univers.\\\\\
On d\'e9fini une r\'e8gle $\\Gamma$ comme ceci : \
\
\\begin\{align*\}\
    \\Gamma : &(E)^n \\longrightarrow E\\\\ \
    &(P_1,P_2,...,P_n) \\longmapsto T\
\\end\{align*\}\
Soit n $\\in$ N. On d\'e9finit l'application $\\Gamma$ par \
\\begin\{align*\}\
    \\forall (P_1,P_2,...,P_n) \\in (E)^n,\\Gamma(P_1,P_2,...,P_n) = P_1\\land P_2 \\land ... \\land P_n = T\
\\end\{align*\}\
\
Avec $T$ la conclusion de la r\'e8gle et $(P_1,P_2,...,P_n)$ les propositions de sa pr\'e9misse.\
% Algorithm Implementation and Development\
\\section\{Algorithm Implementation and Development\}\
Add your algorithm implementation and development here. See Algorithm~\\ref\{alg:example\} for how to include an algorithm in your document. This is how to make an \\textit\{ordered\} list:\
\\begin\{enumerate\}\
    \\item Fluffy swallowed a marble.\
    \\item I took Fluffy to the vet.\
    \\item They took an ultrasound of Fluffy's intestines.\
\\end\{enumerate\}\
\
\\begin\{algorithm\}\
\\begin\{algorithmic\}\
    \\STATE\{Import data from \\texttt\{Testdata.mat\}\}\
    \\FOR\{$j = 1:20$\}\
        \\STATE\{Extract measurement $j$ from \\texttt\{Undata\}\}\
        \\STATE\{Do something useful\}\
    \\ENDFOR\
    \\IF\{$i\\geq 5$\} \
        \\STATE\{$i\\gets i-1$\}\
    \\ELSE\
        \\IF\{$i\\leq 3$\}\
            \\STATE\{$i\\gets i+2$\}\
        \\ENDIF\
    \\ENDIF \
\\end\{algorithmic\}\
\\caption\{Example Algorithm\}\
\\label\{alg:example\}\
\\end\{algorithm\}\
\
% Computational Results\
\\section\{Computational Results\}\
Add your computational results here. See Table~\\ref\{tab:mascots\} for how to include a table in your document. See Figure~\\ref\{fig:dubs\} for how to include figures in your document.\
\
\\begin\{table\}\
    \\centering\
    \\begin\{tabular\}\{rll\}\
    & Name & Years \\\\\
    \\hline\
    1 & Frosty & 1922-1930  \\\\\
    2 & Frosty II & 1930-1936 \\\\\
    3 & Wasky & 1946 \\\\\
    4 & Wasky II & 1947 \\\\\
    5 & Ski & 1954 \\\\\
    6 & Denali & 1958 \\\\\
    7 & King Chinook & 1959-1968\\\\\
    8 & Regent Denali & 1969 \\\\\
    9 & Sundodger Denali & 1981-1992 \\\\\
    10 & King Redoubt & 1992-1998 \\\\\
    11 & Prince Redoubt & 1998 \\\\\
    12 & Spirit & 1999-2008 \\\\\
    13 & Dubs I & 2009-2018 \\\\\
    14 & Dubs II & 2018-Present\
    \\end\{tabular\}\
\
    \\label\{tab:mascots\}\
\\end\{table\}\
\
% begin\{figure\}[tb] % t = top, b = bottom, etc.\
\\begin\{figure\}\
    \\centering\
    \\includegraphics[width=0.5\\linewidth]\{dubs.jpg\}\
    \\caption\{Here is a picture of Dubs \\cite\{webeck_2018\}. Dubs did not swallow a marble.\}\
    \\label\{fig:dubs\}\
\\end\{figure\}\
\
% Summary and Conclusions\
\\section\{Summary and Conclusions\}\
Add your summary and conclusions here.\
\
% References\
\\printbibliography\
\
% Appendices\
\\begin\{appendices\}\
\
% MATLAB Functions\
\\section\{MATLAB Functions\}\
Add your important MATLAB functions here with a brief implementation explanation. This is how to make an \\textbf\{unordered\} list:\
\\begin\{itemize\}\
    \\item \\texttt\{y = linspace(x1,x2,n)\} returns a row vector of \\texttt\{n\} evenly spaced points between \\texttt\{x1\} and \\texttt\{x2\}. \
    \\item \\texttt\{[X,Y] = meshgrid(x,y)\} returns 2-D grid coordinates based on the coordinates contained in the vectors \\texttt\{x\} and \\texttt\{y\}. \\text\{X\} is a matrix where each row is a copy of \\texttt\{x\}, and \\texttt\{Y\} is a matrix where each column is a copy of \\texttt\{y\}. The grid represented by the coordinates \\texttt\{X\} and \\texttt\{Y\} has \\texttt\{length(y)\} rows and \\texttt\{length(x)\} columns.  \
\\end\{itemize\}\
\
% MATLAB Codes\
\\section\{MATLAB Code\}\
Add your MATLAB code here. This section will not be included in your page limit of six pages.\
\
\\begin\{listing\}[h]\
\\inputminted\{matlab\}\{example.m\}\
\\caption\{Example code from external file.\}\
\\label\{listing:examplecode\}\
\\end\{listing\}\
\
\\end\{appendices\}\
\
\\end\{document\}\
}